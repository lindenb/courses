  \documentclass{beamer}
\usepackage[utf8]{inputenc}
\usepackage{hyperref}
\usepackage{graphicx}
\usepackage{listings}
\lstset{frame=single,backgroundcolor=\color{lightgray}}
\usetheme{Warsaw}

\newcommand{\remoteimage}[3]{
\IfFileExists{#1}{}{\immediate\write18{curl -o "#1" "#2"}}
\begin{center}
\includegraphics[#3]{#1}
\end{center}
}

\newcommand{\graphviz}[3]{
\IfFileExists{#1}{}{\immediate\write18{echo #2 | dot -o"#1.png" -Tpng}}
\begin{center}
\includegraphics[#3]{#1.png}
\end{center}
}

\newcommand{\centeredtitle}[1]{
\begin{center}
    \Huge{\bf{#1}}
\end{center}
}

\newcommand{\hugeslide}[1]{
\begin{frame}
\centeredtitle{#1}
\end{frame}
}



\title{Advanced NCBI.\\\href{https://github.com/lindenb/courses}{https://github.com/lindenb/courses}}
\author{Pierre Lindenbaum\\\href{https://twitter.com/yokofakun}{@yokofakun}\\ \href{mailto:plindenbaum@yahoo.fr}{pierre.lindenbaum@univ-nantes.fr}\\ \url{http://plindenbaum.blogspot.com} }\institute{Institut du Thorax. Nantes. France}

\begin{document}



\begin{frame} 
\titlepage
\end{frame}

\begin{frame}
What we will cover:
\end{frame}

\begin{frame}
\remoteimage{jeter01.png}{http://www.ncbi.nlm.nih.gov/corehtml/pmc/pmcgifs/bookshelf/thumbs/th-helpeutils-lrg.png}{}
\end{frame}


 
%%%%%%%%%%%%%%%%%%%%%%%%%%%%%%%%%%%%%%%%%%%
%%
%% FORMAT
%%
%%%%%%%%%%%%%%%%%%%%%%%%%%%%%%%%%%%%%%%%%%%
\hugeslide{Formats}

\begin{frame}[fragile]
\frametitle{Format}
\framesubtitle{Genbank}
\begin{lstlisting}[basicstyle=\tiny,breaklines=false]
LOCUS       X53813                   422 bp    DNA     linear   MAM 22-JUN-1992
DEFINITION  Blue Whale heavy satellite DNA.
ACCESSION   X53813 X17460
VERSION     X53813.1  GI:25
KEYWORDS    satellite DNA.
SOURCE      Balaenoptera musculus (Blue whale)
  ORGANISM  Balaenoptera musculus
            Eukaryota; Metazoa; Chordata; Craniata; Vertebrata; Euteleostomi;
            Mammalia; Eutheria; Laurasiatheria; Cetartiodactyla; Cetacea;
            Mysticeti; Balaenopteridae; Balaenoptera.
REFERENCE   1  (bases 1 to 422)
  AUTHORS   Arnason,U. and Widegren,B.
  TITLE     Composition and chromosomal localization of cetacean highly
            repetitive DNA with special reference to the blue whale,
            Balaenoptera musculus
  JOURNAL   Chromosoma 98 (5), 323-329 (1989)
   PUBMED   2612291
COMMENT     See also <X52700-2> for 1,760 bp common cetacean component clones
            and <X52703-6>,<X53811-4> for the 422 bp heavy satellite clones.
FEATURES             Location/Qualifiers
     source          1..422
                     /organism="Balaenoptera musculus"
                     /mol_type="genomic DNA"
                     /db_xref="taxon:9771"
                     /clone="7"
     misc_feature    1..422
                     /note="heavy satellite DNA"
ORIGIN      
        1 tagttattca acctatccca ctctctagat accccttagc acgtaaagga atattatttg
       61 ggggtccagc catggagaat agtttagaca ctaggatgag ataaggaaca cacccattct
      121 aaagaaatca cattaggatt ctctttttaa gctgttcctt aaaacactag agtcttagaa
      181 atctattgga ggcagaagca gtcaagggta gcctagggtt agggttaggc ttagggttag
      241 ggttagggta cggcttaggg tactgtttcg gggaggggtt caggtacggc gtagggtatg
      301 ggttagggtt agggttaggg ttagtgttag ggttagggct cggtttaggg tacgggttag
      361 gattagggta cgtgttaggg ttagggtagg gcttagggtt agggtacgtg ttagggttag
      421 gg
//
\end{lstlisting}
\end{frame}

\begin{frame}[fragile]
\url{"http://eutils.ncbi.nlm.nih.gov/entrez/eutils/efetch.fcgi?db=nucleotide&id=25}
\frametitle{Format}
\framesubtitle{ASN.1}
\begin{lstlisting}[basicstyle=\tiny,breaklines=false]
Seq-entry ::= seq {
  id {
    embl {
      accession "X53813" ,
      version 1 } ,
    gi 25 } ,
  descr {
    title "Blue Whale heavy satellite DNA" ,
    source {
      org {
        taxname "Balaenoptera musculus" ,
        common "Blue whale" ,
        db {
          {
            db "taxon" ,
            tag
              id 9771 } } ,
        orgname {
          name
            binomial {
              genus "Balaenoptera" ,
              species "musculus" } ,
          lineage "Eukaryota; Metazoa; Chordata; Craniata; Vertebrata;
 Euteleostomi; Mammalia; Eutheria; Laurasiatheria; Cetartiodactyla; Cetacea;
 Mysticeti; Balaenopteridae; Balaenoptera" ,
          gcode 1 ,
          mgcode 2 ,
          div "MAM" } } ,
      subtype {
        {
\end{lstlisting}
\end{frame}


\begin{frame}[fragile]
\frametitle{Format}
\framesubtitle{ASN.1 (schema)}
\url{http://www.ncbi.nlm.nih.gov/data_specs/asn/insdseq.asn}
\begin{lstlisting}[basicstyle=\tiny,breaklines=false]
INSDSeq ::= SEQUENCE {
    locus VisibleString ,
    length INTEGER ,
    strandedness VisibleString OPTIONAL ,
    moltype VisibleString ,
    topology VisibleString OPTIONAL ,
    division VisibleString ,
    update-date VisibleString ,
    create-date VisibleString OPTIONAL ,
    update-release VisibleString OPTIONAL ,
    create-release VisibleString OPTIONAL ,
    definition VisibleString ,
    primary-accession VisibleString OPTIONAL ,
    entry-version VisibleString OPTIONAL ,
    accession-version VisibleString OPTIONAL ,
    other-seqids SEQUENCE OF INSDSeqid OPTIONAL ,
    secondary-accessions SEQUENCE OF INSDSecondary-accn OPTIONAL,
    project VisibleString OPTIONAL ,
    keywords SEQUENCE OF INSDKeyword OPTIONAL ,
    segment VisibleString OPTIONAL ,
    source VisibleString OPTIONAL ,
    organism VisibleString OPTIONAL ,
    taxonomy VisibleString OPTIONAL ,
    references SEQUENCE OF INSDReference OPTIONAL ,
    comment VisibleString OPTIONAL ,
    comment-set SEQUENCE OF INSDComment OPTIONAL ,
    struc-comments SEQUENCE OF INSDStrucComment OPTIONAL ,
    primary VisibleString OPTIONAL ,
    source-db VisibleString OPTIONAL ,
    database-reference VisibleString OPTIONAL ,
    feature-table SEQUENCE OF INSDFeature OPTIONAL ,
    feature-set SEQUENCE OF INSDFeatureSet OPTIONAL ,
    sequence VisibleString OPTIONAL ,  -- Optional for contig, wgs, etc.
    contig VisibleString OPTIONAL ,
    alt-seq SEQUENCE OF INSDAltSeqData OPTIONAL
}
\end{lstlisting}
\end{frame}

\begin{frame}[fragile]
\frametitle{Formats}
\framesubtitle{ASN.1 (tools)}
\begin{center}
DATATOOL
\end{center}
Generate C++ data storage classes based on ASN.1, DTD or XML Schema specification to be used with NCBI data serialization streams.\\
Convert ASN.1 specification into a DTD or XML Schema specification and vice versa.\\
Convert data between ASN.1, XML and JSON formats.\\
\end{frame}

\begin{frame}[fragile]
\frametitle{Formats}
\framesubtitle{XML}
\url{http://eutils.ncbi.nlm.nih.gov/entrez/eutils/efetch.fcgi?db=nucleotide&id=25&retmode=xml}
\begin{lstlisting}[language=xml,basicstyle=\tiny,breaklines=false]
<?xml version="1.0"?>
 <!DOCTYPE GBSet PUBLIC "-//NCBI//NCBI GBSeq/EN" "http://www.ncbi.nlm.nih.gov/dtd/NCBI_GBSeq.dtd">
 <GBSet>
<GBSeq>
  <GBSeq_locus>X53813</GBSeq_locus>
  <GBSeq_length>422</GBSeq_length>
  <GBSeq_strandedness>double</GBSeq_strandedness>
  <GBSeq_moltype>DNA</GBSeq_moltype>
  <GBSeq_topology>linear</GBSeq_topology>
  <GBSeq_division>MAM</GBSeq_division>
  <GBSeq_update-date>22-JUN-1992</GBSeq_update-date>
  <GBSeq_create-date>13-JUL-1990</GBSeq_create-date>
  <GBSeq_definition>Blue Whale heavy satellite DNA</GBSeq_definition>
  <GBSeq_primary-accession>X53813</GBSeq_primary-accession>
  <GBSeq_accession-version>X53813.1</GBSeq_accession-version>
  <GBSeq_other-seqids>
    <GBSeqid>emb|X53813.1|</GBSeqid>
    <GBSeqid>gi|25</GBSeqid>
  </GBSeq_other-seqids>
  <GBSeq_secondary-accessions>
    <GBSecondary-accn>X17460</GBSecondary-accn>
  </GBSeq_secondary-accessions>
  <GBSeq_keywords>
    <GBKeyword>satellite DNA</GBKeyword>
  </GBSeq_keywords>
  <GBSeq_source>Balaenoptera musculus (Blue whale)</GBSeq_source>
  <GBSeq_organism>Balaenoptera musculus</GBSeq_organism>
  <GBSeq_taxonomy>Eukaryota; Metazoa; Chordata; Craniata; Vertebrata; Euteleostomi; Mammalia; Eutheria; Laurasiatheria; Cetartiod
actyla; Cetacea; Mysticeti; Balaenopteridae; Balaenoptera</GBSeq_taxonomy>
  <GBSeq_references>
\end{lstlisting}
\end{frame}


\begin{frame}[fragile]
\frametitle{Formats}
\framesubtitle{XML (DTD)}
\url{http://www.ncbi.nlm.nih.gov/dtd/NCBI_GBSeq.mod.dtd}
\begin{lstlisting}[language=xml,basicstyle=\tiny,breaklines=false]
<!ELEMENT GBSeq (
        GBSeq_locus, 
        GBSeq_length, 
        GBSeq_strandedness?, 
        GBSeq_moltype, 
        GBSeq_topology?, 
        GBSeq_division, 
        GBSeq_update-date, 
        GBSeq_create-date?, 
        GBSeq_update-release?, 
        GBSeq_create-release?, 
        GBSeq_definition, 
        GBSeq_primary-accession?, 
        GBSeq_entry-version?, 
        GBSeq_accession-version?, 
        GBSeq_other-seqids?, 
        GBSeq_secondary-accessions?, 
        GBSeq_project?, 
        GBSeq_keywords?, 
        GBSeq_segment?, 
        GBSeq_source?, 
        GBSeq_organism?, 
        GBSeq_taxonomy?, 
        GBSeq_references?, 
        GBSeq_comment?, 
        GBSeq_comment-set?, 
        GBSeq_struc-comments?, 
        (...)
\end{lstlisting}
\end{frame}




\begin{frame} 
 The E-Utilities are a set of seven server-side programs that provide a stable interface to the search, retrieval, and linking functions of the Entrez system, using a fixed URL syntax. The output provided by the E-Utilities is in XML format, with the notable exception of the EFetch utility, which returns data in a variety of formats. The E-Utilities are designed to be called from within a computer program that can process their output. Calling an E-Utility from any of the common programming languages— including Perl, Python, and Java—is a simple matter of posting a URL.
\end{frame}

%%%%%%%%%%%%%%%%%%%%%%%%%%%%%%%%%%%%%%%%%%%
%%
%% EINFO
%%
%%%%%%%%%%%%%%%%%%%%%%%%%%%%%%%%%%%%%%%%%%%

\hugeslide{EInfo}

\begin{frame}[fragile]
\frametitle{EInfo}
Provides a list of the names of all valid Entrez databases.\\
Provides statistics for a single database, including lists of indexing fields and available link names.
\end{frame}

\begin{frame}[fragile]
\frametitle{EInfo}
\small
\url{http://eutils.ncbi.nlm.nih.gov/entrez/eutils/einfo.fcgi}
\end{frame}



\begin{frame}[fragile]
\frametitle{EInfo}
\framesubtitle{XML Ouput}

\small
\begin{lstlisting}[language=xml]
<!DOCTYPE eInfoResult PUBLIC "-//NLM//DTD eInfoResult, 11 May 2002//EN" "http://www.ncbi.nlm.nih.gov/entrez/query/DTD/eInfo_020511.dtd">
<eInfoResult>
  <DbList>
    <DbName>pubmed</DbName>
    <DbName>protein</DbName>
    <DbName>nuccore</DbName>
    <DbName>nucleotide</DbName>
    <DbName>nucgss</DbName>
    <DbName>nucest</DbName>
    <DbName>structure</DbName>
    <DbName>genome</DbName>
    <DbName>assembly</DbName>
    <DbName>gcassembly</DbName>
    <DbName>genomeprj</DbName>
    <DbName>bioproject</DbName>
    <DbName>biosample</DbName>
    <DbName>biosystems</DbName>
    <DbName>blastdbinfo</DbName>
    <DbName>books</DbName>
    <DbName>cdd</DbName>
    <DbName>clinvar</DbName>
 (...)
\end{lstlisting}
\end{frame}


\begin{frame}[fragile]
\frametitle{EInfo}
Return statistics for Entrez database:\\
\small
\url{http://eutils.ncbi.nlm.nih.gov/entrez/eutils/einfo.fcgi?db=DbName}
\end{frame}


\begin{frame}[fragile]
\frametitle{EInfo}
\framesubtitle{Statistics for Pubmed}
\url{http://eutils.ncbi.nlm.nih.gov/entrez/eutils/einfo.fcgi?db=pubmed}
\begin{lstlisting}[language=xml,basicstyle=\tiny,breaklines=false]
<?xml version="1.0"?>
<!DOCTYPE eInfoResult PUBLIC "-//NLM//DTD eInfoResult, 11 May 2002//EN" "http://www.ncbi.nlm.nih.gov/entrez/query/DTD/eInfo_020511.dtd">
<eInfoResult>
  <DbInfo>
    <DbName>pubmed</DbName>
    <MenuName>PubMed</MenuName>
    <Description>PubMed bibliographic record</Description>
    <DbBuild>Build130805-2117m.4</DbBuild>
    <Count>22974581</Count>
    <LastUpdate>2013/08/06 08:33</LastUpdate>
    <FieldList>
      (...)
      <Field>
        <Name>UID</Name>
        <FullName>UID</FullName>
        <Description>Unique number assigned to publication</Description>
        <TermCount>0</TermCount>
        <IsDate>N</IsDate>
        <IsNumerical>Y</IsNumerical>
        <SingleToken>Y</SingleToken>
        <Hierarchy>N</Hierarchy>
        <IsHidden>Y</IsHidden>
      </Field>
      <Field>
(...)
\end{lstlisting}
\end{frame}

\begin{frame}[fragile]
\frametitle{EInfo}
\framesubtitle{Statistics for Pubmed}
\url{http://eutils.ncbi.nlm.nih.gov/entrez/eutils/einfo.fcgi?db=pubmed}
\begin{lstlisting}[language=xml,basicstyle=\tiny,breaklines=false]
(...)
      <Link>
        <Name>pubmed_taxonomy_entrez</Name>
        <Menu>Taxonomy via GenBank</Menu>
        <Description>Taxonomy records associated with the current articles through taxonomic information on related molecular database records (Nucleotide, Protein, Gene, SNP, Structu
re).</Description>
        <DbTo>taxonomy</DbTo>
      </Link>
      <Link>
        <Name>pubmed_unigene</Name>
        <Menu>UniGene Links</Menu>
        <Description>UniGene clusters of expressed sequences that are associated with the current articles through references on the clustered sequence records and related Gene record
s.</Description>
        <DbTo>unigene</DbTo>
      </Link>
      <Link>
        <Name>pubmed_unists</Name>
        <Menu>UniSTS Links</Menu>
        <Description>Genetic, physical, and sequence mapping reagents in the UniSTS database associated with the current articles through references on sequence tagged site (STS) subm
issions as well as automated searching of PubMed abstracts and full-text PubMed Central articles for marker names.</Description>
        <DbTo>unists</DbTo>
      </Link>
    </LinkList>
  </DbInfo>
</eInfoResult>
\end{lstlisting}
\end{frame}



%%%%%%%%%%%%%%%%%%%%%%%%%%%%%%%%%%%%%%%%%%%
%%
%% ESearch
%%
%%%%%%%%%%%%%%%%%%%%%%%%%%%%%%%%%%%%%%%%%%%

\hugeslide{ESearch}

\begin{frame}[fragile]
\frametitle{ESearch}
\begin{itemize}
\item Provides a list of UIDs matching a text query
\item Posts the results of a search on the History server
\item Downloads all UIDs from a dataset stored on the History server
\item Combines or limits UID datasets stored on the History server
\item Sorts sets of UIDs
\end{itemize}
\end{frame}


\begin{frame}[fragile]
\frametitle{ESearch}
\framesubtitle{Syntax}
\begin{lstlisting}
http://eutils.ncbi.nlm.nih.gov/entrez/eutils/esearch.fcgi
\end{lstlisting}
\end{frame}

\begin{frame}[fragile]
\frametitle{ESearch}
\framesubtitle{Searching }
\begin{lstlisting}[language=bash,basicstyle=\tiny,breaklines=true]
curl  "http://eutils.ncbi.nlm.nih.gov/entrez/eutils/esearch.fcgi?db=nucleotide&term=%22Mammuthus%20primigenius%22%5BORGN%5D" |\
xmllint --format -
\end{lstlisting}

\begin{lstlisting}[language=xml,basicstyle=\tiny,breaklines=false]
<eSearchResult>
  <Count>684</Count>
  <RetMax>20</RetMax>
  <RetStart>0</RetStart>
  <IdList>
    <Id>507866428</Id>
    <Id>124056416</Id>
    <Id>383843869</Id>
    <Id>383843867</Id>
    <Id>383843865</Id>
    <Id>383843863</Id>
    <Id>383843861</Id>
    <Id>383843859</Id>
    <Id>383843857</Id>
    <Id>383843855</Id>
    <Id>383843853</Id>
    <Id>383843851</Id>
    <Id>383843849</Id>
    <Id>383843847</Id>
    <Id>383843845</Id>
    <Id>157367690</Id>
    <Id>157367676</Id>
    <Id>157367662</Id>
    <Id>157367648</Id>
    <Id>157367634</Id>
  </IdList>
  <TranslationSet>
    <Translation>
      <From>"Mammuthus primigenius"[ORGN]</From>
      <To>"Mammuthus primigenius"[Organism]</To>
    </Translation>
  </TranslationSet>
  <TranslationStack>
    <TermSet>
      <Term>"Mammuthus primigenius"[Organism]</Term>
      <Field>Organism</Field>
      <Count>684</Count>
      <Explode>Y</Explode>
    </TermSet>
    <OP>GROUP</OP>
  </TranslationStack>
  <QueryTranslation>"Mammuthus primigenius"[Organism]</QueryTranslation>
</eSearchResult>
\end{lstlisting}
\end{frame}

\begin{frame}[fragile]
\frametitle{ESearch}
\framesubtitle{retmax  }
\begin{lstlisting}[language=bash,basicstyle=\tiny,breaklines=true]
curl  "http://eutils.ncbi.nlm.nih.gov/entrez/eutils/esearch.fcgi?db=nucleotide&term=%22Mammuthus%20primigenius%22%5BORGN%5D&retmax=2" |\
xmllint --format -
\end{lstlisting}

\begin{lstlisting}[language=xml,basicstyle=\tiny,breaklines=false]
<eSearchResult>
  <Count>684</Count>
  <RetMax>2</RetMax>
  <RetStart>0</RetStart>
  <IdList>
    <Id>507866428</Id>
    <Id>124056416</Id>
  </IdList>
  <TranslationSet>
    <Translation>
      <From>"Mammuthus primigenius"[ORGN]</From>
      <To>"Mammuthus primigenius"[Organism]</To>
    </Translation>
  </TranslationSet>
  <TranslationStack>
    <TermSet>
      <Term>"Mammuthus primigenius"[Organism]</Term>
      <Field>Organism</Field>
      <Count>684</Count>
      <Explode>Y</Explode>
    </TermSet>
    <OP>GROUP</OP>
  </TranslationStack>
  <QueryTranslation>"Mammuthus primigenius"[Organism]</QueryTranslation>
</eSearchResult>
\end{lstlisting}
\end{frame}


\begin{frame}[fragile]
\frametitle{ESearch}
\framesubtitle{retstart  }
\begin{lstlisting}[language=bash,basicstyle=\tiny,breaklines=true]
curl  "http://eutils.ncbi.nlm.nih.gov/entrez/eutils/esearch.fcgi?db=nucleotide&term=%22Mammuthus%20primigenius%22%5BORGN%5D&retmax=2&retstart=100" |\
xmllint --format -
\end{lstlisting}

\begin{lstlisting}[language=xml,basicstyle=\tiny,breaklines=false]
<eSearchResult>
  <Count>684</Count>
  <RetMax>3</RetMax>
  <RetStart>100</RetStart>
  <IdList>
    <Id>300810656</Id>
    <Id>300810655</Id>
    <Id>300810654</Id>
  </IdList>
  <TranslationSet>
    <Translation>
      <From>"Mammuthus primigenius"[ORGN]</From>
      <To>"Mammuthus primigenius"[Organism]</To>
    </Translation>
  </TranslationSet>
  <TranslationStack>
    <TermSet>
      <Term>"Mammuthus primigenius"[Organism]</Term>
      <Field>Organism</Field>
      <Count>684</Count>
      <Explode>Y</Explode>
    </TermSet>
    <OP>GROUP</OP>
  </TranslationStack>
  <QueryTranslation>"Mammuthus primigenius"[Organism]</QueryTranslation>
</eSearchResult>
\end{lstlisting}
\end{frame}

\begin{frame}[fragile]
\frametitle{ESearch}
\framesubtitle{retcount  }
\begin{lstlisting}[language=bash,basicstyle=\tiny,breaklines=true]
curl  "http://eutils.ncbi.nlm.nih.gov/entrez/eutils/esearch.fcgi?db=nucleotide&term=%22Mammuthus%20primigenius%22%5BORGN%5D&rettype=count" |\
xmllint --format -
\end{lstlisting}

\begin{lstlisting}[language=xml,breaklines=false]
<eSearchResult>
  <Count>684</Count>
</eSearchResult>
\end{lstlisting}
\end{frame}

%%%%%%%%%%%%%%%%%%%%%%%%%%%%%%%%%%%%%%%%%%%
%%
%% ESummary
%%
%%%%%%%%%%%%%%%%%%%%%%%%%%%%%%%%%%%%%%%%%%%

\hugeslide{ESummary}


\begin{frame}[fragile]
\frametitle{ESummary}
\framesubtitle{Syntax}
\begin{lstlisting}
http://eutils.ncbi.nlm.nih.gov/entrez/eutils/esummary.fcgi?db=(DB)&id=(TERM)
\end{lstlisting}
\end{frame}


\begin{frame}[fragile]
\frametitle{ESummary}
\begin{lstlisting}[language=bash,basicstyle=\tiny,breaklines=true]
$ curl -s  "http://eutils.ncbi.nlm.nih.gov/entrez/eutils/esummary.fcgi?db=nucleotide&id=507866428"
\end{lstlisting}

\begin{lstlisting}[language=xml,basicstyle=\tiny,breaklines=false]
<eSummaryResult>
<DocSum>
	<Id>507866428</Id>
	<Item Name="Caption" Type="String">KC524742</Item>
	<Item Name="Title" Type="String">Mammuthus primigenius isolate CME2005/915 myoglobin (Mb) gene, partial cds</Item>
	<Item Name="Extra" Type="String">gi|507866428|gb|KC524742.1|[507866428]</Item>
	<Item Name="Gi" Type="Integer">507866428</Item>
	<Item Name="CreateDate" Type="String">2013/06/15</Item>
	<Item Name="UpdateDate" Type="String">2013/06/21</Item>
	<Item Name="Flags" Type="Integer">0</Item>
	<Item Name="TaxId" Type="Integer">37349</Item>
	<Item Name="Length" Type="Integer">9042</Item>
	<Item Name="Status" Type="String">live</Item>
	<Item Name="ReplacedBy" Type="String"></Item>
	<Item Name="Comment" Type="String"><![CDATA[  ]]></Item>
</DocSum>
</eSummaryResult>
\end{lstlisting}
\end{frame}

\begin{frame}[fragile]
\frametitle{ESummary}
\begin{lstlisting}[language=bash,basicstyle=\tiny,breaklines=true]
$ curl -s  "http://eutils.ncbi.nlm.nih.gov/entrez/eutils/esummary.fcgi?db=snp&id=25"
\end{lstlisting}

\begin{lstlisting}[language=xml,basicstyle=\tiny,breaklines=false]
<eSummaryResult>
<DocSum>
	<Id>25</Id>
	<Item Name="SNP_ID" Type="Integer">25</Item>
	<Item Name="Organism" Type="String"></Item>
	<Item Name="ALLELE_ORIGIN" Type="String"></Item>
	<Item Name="GLOBAL_MAF" Type="String">0.4913</Item>
	<Item Name="GLOBAL_POPULATION" Type="String"></Item>
	<Item Name="GLOBAL_SAMPLESIZE" Type="Integer">0</Item>
	<Item Name="SUSPECTED" Type="String"></Item>
	<Item Name="CLINICAL_SIGNIFICANCE" Type="String"></Item>
	<Item Name="GENE" Type="String">THSD7A</Item>
	<Item Name="LOCUS_ID" Type="Integer">221981</Item>
	<Item Name="ACC" Type="String">NM_015204.2,NT_007819.17</Item>
	<Item Name="CHR" Type="String">7</Item>
	<Item Name="WEIGHT" Type="Integer">1</Item>
	<Item Name="HANDLE" Type="String">1000GENOMES,BGI,BL,BUSHMAN,COMPLETE_GENOMICS,CSHL-HAPMAP,GMI,ILLUMINA-UK,KWOK,PERLEGEN,SSMP,TISHKOFF</Item>
	<Item Name="FXN_CLASS" Type="String">intron-variant</Item>
	<Item Name="VALIDATED" Type="String">by-1000G,by-cluster,by-frequency,by-hapmap</Item>
	<Item Name="GTYPE" Type="String">true</Item>
	<Item Name="NONREF" Type="String">false</Item>
	<Item Name="DOCSUM" Type="String">HGVS=NC_000007.13:g.11584142T&gt;C,NG_027670.1:g.292683A&gt;G,NM_015204.2:c.1454-1398A&gt;G,NT_007819.17:g.11574142T&gt;C|SEQ=TCTGTGAGCTTCTGCATGCAATCCT[A/G]TGCAATTGGAATTTGATAGTCCTTT|GENE=THSD7A:221981</Item>
	<Item Name="HET" Type="Integer">50</Item>
	<Item Name="SRATE" Type="Integer">0</Item>
	<Item Name="TAX_ID" Type="Integer">9606</Item>
	<Item Name="CHRRPT" Type="String">25|2|0|1|1|1|7|NT_007819.17|11574141|11584142|THSD7A|0.499848|0.00872267| |51|1|1|36|138|0| | |T:2178:0.4913</Item>
	<Item Name="ORIG_BUILD" Type="Integer">36</Item>
	<Item Name="UPD_BUILD" Type="Integer">138</Item>
	<Item Name="CREATEDATE" Type="String">2000-09-19 17:02</Item>
	<Item Name="UPDATEDATE" Type="String">2013-06-21 14:17</Item>
	<Item Name="POP_CLASS" Type="String"></Item>
	<Item Name="METHOD_CLASS" Type="String">computed,hybridize,sequence,unknown</Item>
	<Item Name="SNP3D" Type="String"></Item>
	<Item Name="LINKOUT" Type="String">ILLUMINA-UK|http://www.illumina.com/HumanGenomeNA18507_000019106_NCBI36.1_chr7_11550667</Item>
	<Item Name="SS" Type="Integer">654151077</Item>
	<Item Name="LOCSNPID" Type="String">7_11584142</Item>
	<Item Name="ALLELE" Type="String">R</Item>
	<Item Name="SNP_CLASS" Type="String">snp</Item>
	<Item Name="CHRPOS" Type="String">7:11584142</Item>
	<Item Name="CONTIGPOS" Type="String">NT_007819.17:11574142</Item>
	<Item Name="TEXT" Type="String"></Item>
	<Item Name="LOOKUP" Type="String">325952</Item>
</DocSum>
</eSummaryResult>
\end{lstlisting}
\end{frame}


\begin{frame}[fragile]
\frametitle{ESummary}
\begin{lstlisting}[language=bash,basicstyle=\tiny,breaklines=true]
$ curl -s  "http://eutils.ncbi.nlm.nih.gov/entrez/eutils/esummary.fcgi?db=pubmed&id=7939126"
\end{lstlisting}

\begin{lstlisting}[language=xml,basicstyle=\tiny,breaklines=false]
<eSummaryResult>
<DocSum>
	<Id>7939126</Id>
	<Item Name="PubDate" Type="Date">1994 Apr</Item>
	<Item Name="EPubDate" Type="Date"></Item>
	<Item Name="Source" Type="String">Sleep</Item>
	<Item Name="AuthorList" Type="List">
		<Item Name="Author" Type="String">Broughton R</Item>
		<Item Name="Author" Type="String">Billings R</Item>
		<Item Name="Author" Type="String">Cartwright R</Item>
		<Item Name="Author" Type="String">Doucette D</Item>
		<Item Name="Author" Type="String">Edmeads J</Item>
		<Item Name="Author" Type="String">Edwardh M</Item>
		<Item Name="Author" Type="String">Ervin F</Item>
		<Item Name="Author" Type="String">Orchard B</Item>
		<Item Name="Author" Type="String">Hill R</Item>
		<Item Name="Author" Type="String">Turrell G</Item>
	</Item>
	<Item Name="LastAuthor" Type="String">Turrell G</Item>
	<Item Name="Title" Type="String">Homicidal somnambulism: a case report.</Item>
	<Item Name="Volume" Type="String">17</Item>
	<Item Name="Issue" Type="String">3</Item>
	<Item Name="Pages" Type="String">253-64</Item>
	<Item Name="LangList" Type="List">
		<Item Name="Lang" Type="String">English</Item>
	</Item>
	<Item Name="NlmUniqueID" Type="String">7809084</Item>
	<Item Name="ISSN" Type="String">0161-8105</Item>
	<Item Name="ESSN" Type="String">1550-9109</Item>
	<Item Name="PubTypeList" Type="List">
		<Item Name="PubType" Type="String">Journal Article</Item>
	</Item>
	<Item Name="RecordStatus" Type="String">PubMed - indexed for MEDLINE</Item>
	<Item Name="PubStatus" Type="String">ppublish</Item>
	<Item Name="ArticleIds" Type="List">
		<Item Name="pubmed" Type="String">7939126</Item>
		<Item Name="eid" Type="String">7939126</Item>
		<Item Name="rid" Type="String">7939126</Item>
	</Item>
	<Item Name="History" Type="List">
		<Item Name="pubmed" Type="Date">1994/04/01 00:00</Item>
		<Item Name="medline" Type="Date">1994/04/01 00:01</Item>
		<Item Name="entrez" Type="Date">1994/04/01 00:00</Item>
	</Item>
	<Item Name="References" Type="List"></Item>
	<Item Name="HasAbstract" Type="Integer">1</Item>
	<Item Name="PmcRefCount" Type="Integer">4</Item>
	<Item Name="FullJournalName" Type="String">Sleep</Item>
	<Item Name="ELocationID" Type="String"></Item>
	<Item Name="SO" Type="String">1994 Apr;17(3):253-64</Item>
</DocSum>
</eSummaryResult>
\end{lstlisting}
\end{frame}


%%%%%%%%%%%%%%%%%%%%%%%%%%%%%%%%%%%%%%%%%%%
%%
%% EFetch
%%
%%%%%%%%%%%%%%%%%%%%%%%%%%%%%%%%%%%%%%%%%%%

\hugeslide{EFetch}


\begin{frame}[fragile]
\frametitle{EFetch}
\framesubtitle{Syntax}
\begin{lstlisting}
$ curl -s  "http://eutils.ncbi.nlm.nih.gov/entrez/eutils/efetch.fcgi?db=(db)&id=(ID)" 
\end{lstlisting}
\end{frame}

\begin{frame}[fragile]
\frametitle{EFetch}
Default 
\url{http://eutils.ncbi.nlm.nih.gov/entrez/eutils/efetch.fcgi?db=nucleotide&id=507866428}
\begin{lstlisting}[language=bash,basicstyle=\tiny,breaklines=true]
Seq-entry ::= set {
  class nuc-prot ,
  descr {
    source {
      genome genomic ,
      org {
        taxname "Mammuthus primigenius" ,
        common "woolly mammoth" ,
        db {
          {
            db "taxon" ,
            tag
              id 37349 } } ,
        orgname {
          name
            binomial {
              genus "Mammuthus" ,
              species "primigenius" } ,
          mod {
            {

\end{lstlisting}
\end{frame}

\begin{frame}[fragile]
\frametitle{EFetch}
\url{http://eutils.ncbi.nlm.nih.gov/entrez/eutils/efetch.fcgi?db=nucleotide&id=507866428&rettype=fasta}
\begin{lstlisting}[basicstyle=\tiny,breaklines=true]
>gi|507866428|gb|KC524742.1| Mammuthus primigenius isolate CME2005/915 myoglobin (Mb) gene, partial cds
GCACTTGCTTTTTTTGTCTTCTTCAGACCACGACATGGGACTCAGCGACGGGGAATGGGAGTTGGTGTTG
AAAACCTGGGGGAAAGTGGAGGCTGACATCCCGGGCCATGGGCTGGAAGTCTTCGTCAGGTAAAGGAAGA
AATCCTGTGGCCCCCATCACCCACCCCNNNNNNNNNNNNNNNNNNNNNNNNNNNNNNNNNNNNNNNNNNN
NNNNNNNNNNNNNNNNNNNNNNNNNNNNNNNNNNNNNNNNNNNNNNNNNNNNNNNNNNNNNNNNNNNNNN
NNNNNNNNNNNNNNNNNNNNNNNNNNNNNNNNNNNNNNNNNNNNNNNNNNNNNNNNNNNNNNNNNNNNNN
NNNNNNNNNNNNNNNNNNNNNNNNNNNNNNNNNNNNNNNNNNNNNNNNNNNNNNNNNNNNNNNNNNNNNN
NNNNNNNNNNNNNNNNNNNNNNNNNNNNNNNNNNNNNNNNNNNNNNNNNNNNNNNNNNNNNNNNNNNNNN
NNNNNNNNNNNNNNNNNNNNNNNNNNNNNNNNNNNNNNNNNNNNNNNNNNNNNNNNNNNNNNNNNNNNNN
NNNNNNNNNNNNNNNNNNNNNNNNNNNNNNNNNNNNNNNNNNNNNNNNNNNNNNNNNNNNNNNNNNNNNN
\end{lstlisting}
\end{frame}

\begin{frame}[fragile]
\frametitle{EFetch}
\url{http://eutils.ncbi.nlm.nih.gov/entrez/eutils/efetch.fcgi?db=nucleotide&id=507866428&rettype=fasta&retmode=xml}
\begin{lstlisting}[language=xml,basicstyle=\tiny,breaklines=false]
<?xml version="1.0"?>
 <!DOCTYPE TSeqSet PUBLIC "-//NCBI//NCBI TSeq/EN" 
 <TSeqSet>
<TSeq>
  <TSeq_seqtype value="nucleotide"/>
  <TSeq_gi>507866428</TSeq_gi>
  <TSeq_accver>KC524742.1</TSeq_accver>
  <TSeq_taxid>37349</TSeq_taxid>
  <TSeq_orgname>Mammuthus primigenius</TSeq_orgnam
  <TSeq_defline>Mammuthus primigenius isolate CME2
  <TSeq_length>9042</TSeq_length>
  <TSeq_sequence>GCACTTGCTTTTTTTGTCTTCTTCAGACCACGA
</TSeq>
</TSeqSet>
\end{lstlisting}
\end{frame}


\begin{frame}[fragile]
\frametitle{EFetch}
\url{http://eutils.ncbi.nlm.nih.gov/entrez/eutils/efetch.fcgi?db=nucleotide&id=507866428&retmode=xml}
\begin{lstlisting}[language=xml,basicstyle=\tiny,breaklines=false]
<GBSeq>
  <GBSeq_locus>KC524742</GBSeq_locus>
  <GBSeq_length>9042</GBSeq_length>
  <GBSeq_strandedness>double</GBSeq_strandedness>
  <GBSeq_moltype>DNA</GBSeq_moltype>
  <GBSeq_topology>linear</GBSeq_topology>
  <GBSeq_division>MAM</GBSeq_division>
  <GBSeq_update-date>21-JUN-2013</GBSeq_update-date>
  <GBSeq_create-date>15-JUN-2013</GBSeq_create-date>
  <GBSeq_definition>Mammuthus primigenius isolate CME2005/915 myoglobin (Mb) gene, partial cds</GBSeq_definition>
  <GBSeq_primary-accession>KC524742</GBSeq_primary-accession>
  <GBSeq_accession-version>KC524742.1</GBSeq_accession-version>
  <GBSeq_other-seqids>
    <GBSeqid>gb|KC524742.1|</GBSeqid>
    <GBSeqid>gi|507866428</GBSeqid>
  </GBSeq_other-seqids>
  <GBSeq_source>Mammuthus primigenius (woolly mammoth)</GBSeq_source>
  <GBSeq_organism>Mammuthus primigenius</GBSeq_organism>
(...)
\end{lstlisting}
\end{frame}

\begin{frame}[fragile]
\frametitle{EFetch}
\url{http://eutils.ncbi.nlm.nih.gov/entrez/eutils/efetch.fcgi?db=nucleotide&id=507866428&rettype=gb}
\begin{lstlisting}[language=bash,basicstyle=\tiny,breaklines=false]
LOCUS       KC524742                9042 bp    DNA     linear   MAM 21-JUN-2013
DEFINITION  Mammuthus primigenius isolate CME2005/915 myoglobin (Mb) gene,
            partial cds.
ACCESSION   KC524742
VERSION     KC524742.1  GI:507866428
KEYWORDS    .
SOURCE      Mammuthus primigenius (woolly mammoth)
  ORGANISM  Mammuthus primigenius
            Eukaryota; Metazoa; Chordata; Craniata; Vertebrata; Euteleostomi;
            Mammalia; Eutheria; Afrotheria; Proboscidea; Elephantidae;
            Mammuthus.
REFERENCE   1  (bases 1 to 9042)
  AUTHORS   Mirceta,S., Signore,A.V., Burns,J.M., Cossins,A.R., Campbell,K.L.
            and Berenbrink,M.
  TITLE     Evolution of mammalian diving capacity traced by myoglobin net
            surface charge
  JOURNAL   Science 340 (6138), 1234192 (2013)
   PUBMED   23766330
REFERENCE   2  (bases 1 to 9042)
  AUTHORS   Signore,A.V., Campbell,K.L. and Poinar,H.N.
  TITLE     Direct Submission
  JOURNAL   Submitted (09-JAN-2013) Biological Sciences, University of
            Manitoba, 50 Sifton Road, Winnipeg, Manitoba R3T2N2, Canada
COMMENT     ##Assembly-Data-START##
            Sequencing Technology :: Sanger dideoxy sequencing
            ##Assembly-Data-END##
FEATURES             Location/Qualifiers
     source          1..9042
                     /organism="Mammuthus primigenius"
                     /mol_type="genomic DNA"
                     /isolate="CME2005/915"
                     /db_xref="taxon:37349"
                     /tissue_type="bone"
     gene            <35..>9042
                     /gene="Mb"
     mRNA            join(<35..129,5627..5849,8979..>9042)
                     /gene="Mb"
                     /product="myoglobin"
     CDS             join(35..129,5627..5849,8979..>9042)

\end{lstlisting}
\end{frame}


\end{document}

