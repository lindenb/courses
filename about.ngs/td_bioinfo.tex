\documentclass{article}
\usepackage[utf8]{inputenc}
\usepackage{hyperref}
\usepackage{graphicx}
\usepackage{listings}
\usepackage{xcolor}
\lstset{frame=single,backgroundcolor=\color{lightgray},numbers=left,breaklines=true,basicstyle=\ttfamily}


\title{TD Bioinfo}
\author{Pierre Lindenbaum - Institut du Thorax. Nantes. France}
\date{\today}

\begin{document}
\maketitle
\begin{abstract}
A few questions about
A basic linux commands are required, no external program should be installed.
\end{abstract}

\section{Linux}

What's the basic function of the following linux commands ?
\begin{itemize}
\item cd
\item ls
\item pwd
\item ls
\item cp
\item rm
\item mv
\item cat
\item more
\item grep
\item sort
\item uniq
\item paste
\item join
\item tr
\item head
\item tail
\item mkdir
\item awk
\end{itemize}


\section{A few questions}

What is a 'SNP' ? 

What is a 'Genome build' ?

What's the size of the human genome ?

how many chromosomes ?

what the size of the longest human chromosome ?

What's the difference between a binary file and flat file ?

How many bytes do you need to store the *length* of the human genome ?

In sequence containing the only following letters, how many bases can you store in one byte ?

What the advantage of storing data in a tab delimited format vs using a structured format (e.g: XML, JSON, ASN.1) ?

\section{Bash}

What's the purpose of the following bash command line ? ( https://gist.github.com/lindenb/65e98e5752ea26eb9868 )

\begin{lstlisting}[language=bash]
$ curl  "http://hgdownload.cse.ucsc.edu/goldenPath/hg19/chromosomes/chrM.fa.gz" |\
gunzip -c |\
tail -n +2 |\
tr "[:lower:]" "[:upper:]" |\
tr -d '\n' |\
tr "AC" "TG" |\
sed 's/\(.\)/\1#/g' |\
tr "#" "\n" |\
LC_ALL=C sort |\
uniq -c |\
sed 's/^[ ]*//' |\
cut -d ' ' -f1 |\
tr "\n" " " |\
awk '{printf("%f %%\n",$1/($1+$2));}' > result.txt
\end{lstlisting}

\section{Makefile}
(wikipedia:) "In software development, Make is a utility that automatically builds executable programs and libraries from source code by reading files called makefiles which specify how to derive the target program". You'll find many simple tutorials for make over the web.

The following Makefile creates a tab-delimited file 'gene\_ontology\_map.txt' containing the following column: gene-name(hgnc\_symbol), chromosome, start\_position, end\_position and gene-ontology term.

Complete the following Makefile by replacing the ???????????? with the correct pipeline ( https://gist.github.com/lindenb/65e98e5752ea26eb9868 ).

\begin{lstlisting}[language=make]
.PHONY: all clean

#
# this is the top default target
#
all: gene_ontology_map.txt

# using the 'join' command
# join gene_and_position.bed and gene_and_go.txt to create 
# a file with the following columns:  
# hgnc_symbol, chromosome, start_position, end_position and GO-ID
gene_ontology_map.txt: gene_and_position.tsv gene_and_go.txt
	join ??????????????????????????????????

# using a web browser: go on biomart http://www.ensembl.org/biomart/martview/
# select 'ensembl Gene'
# select 'dataset: homo sapiens'
# select the following attributes : 
#	Chromosome Name
#	HGNC symbol 
#	Gene Start (bp)
#	Gene End (bp)
# 'show query in XML format' and use it in the curl command below
# keep the lines having a HGNC symbol
# sort on HGNC symbol 
gene_and_position.tsv : 
	curl --form-string query='<?xml version="1.0" ??????????????????????????????????
#
# remove the headers from gene_association.goa_human
# keep the columns DB_Object_Symbol (3) and GO_ID (5)
# and sort on DB_Object_Symbol (=HGNC symbol )
# and remove the duplicated lines
gene_and_go.txt : gene_association.goa_human
	grep -v '^!' $<  | ??????????????????????????????????


#download GO-association for human
#format described in http://www.geneontology.org/GO.format.gaf-1_0.shtml
gene_association.goa_human:
	curl -s "http://cvsweb.geneontology.org/cgi-bin/cvsweb.cgi/go/gene-associations/$@.gz?rev=HEAD" |\
		gunzip -c > $@

#
# cleanup everything
#
clean:
	rm -f gene_association.goa_human gene_and_go.txt  \
		gene_and_position.tsv \
		gene_ontology_map.txt
\end{lstlisting}


\section{The Human Genome}

the chromosome for the latest build are available at:

http://hgdownload.cse.ucsc.edu/goldenPath/hg19/chromosomes/

what's the size of the chr22 ?

look at the 100000 first lines of the chr22. Explain what you see.

get a count of each base in the chr22

using 'tr' and 'rev', get the reverse complement of chrM.



\end{document}
