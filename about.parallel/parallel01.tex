\documentclass{article}
\usepackage[utf8]{inputenc}
\usepackage{hyperref}
\usepackage{graphicx}
\usepackage{listings}
\usepackage{amssymb}
\usepackage{color}
\usepackage{xcolor}
\usepackage{indentfirst}
\usepackage{makeidx}

\makeindex


\lstset{language=bash,frame=single,backgroundcolor=\color{lightgray},numbers=left,breaklines=true,breakautoindent=true,basicstyle=\small}
\date{\today}
\title{GNU Parallel for Bioinformatics}
\author{Pierre Lindenbaum\\\href{https://twitter.com/yokofakun}{@yokofakun}\\\url{http://plindenbaum.blogspot.com} }


\def\prl{\textbf{parallel}}
\def\bam{\textbf{bam}}
\def\BAM{\textbf{BAM}}
\def\samtools{\textbf{samtools}}

\begin{document}
\maketitle
\begin{abstract}
This document follows the Ole Tange's \prl{} tutorial \url{http://www.gnu.org/software/parallel/parallel_tutorial.html}.
\end{abstract}

\tableofcontents

\section{Input Source}
\subsection{A single input source}
\subsubsection{Input can be read from the command line.}
\begin{lstlisting}
$ parallel file ::: samtools-0.1.18/examples/*.bam
\end{lstlisting}
output:
\begin{lstlisting}
samtools-0.1.18/examples/ex1a.bam: gzip compressed data, extra field
samtools-0.1.18/examples/ex1.bam: gzip compressed data, extra field
samtools-0.1.18/examples/ex1b.bam: gzip compressed data, extra field
samtools-0.1.18/examples/ex1f.bam: gzip compressed data, extra field
samtools-0.1.18/examples/ex1f-rmduppe.bam: gzip compressed data, extra field
samtools-0.1.18/examples/ex1f-rmdupse.bam: gzip compressed data, extra field
samtools-0.1.18/examples/ex1_sorted.bam: gzip compressed data, extra field
samtools-0.1.18/examples/toy.bam: gzip compressed data, extra field
\end{lstlisting}

\subsubsection{The input source can be a file}
\begin{lstlisting}
$ find samtools-0.1.18/examples/ -name "*.bam" -type f > listbams.txt
$ parallel -a listbams.txt file
\end{lstlisting}
output:
\begin{lstlisting}
samtools-0.1.18/examples/ex1a.bam: gzip compressed data, extra field
samtools-0.1.18/examples/ex1.bam: gzip compressed data, extra field
samtools-0.1.18/examples/ex1b.bam: gzip compressed data, extra field
samtools-0.1.18/examples/ex1f.bam: gzip compressed data, extra field
samtools-0.1.18/examples/ex1f-rmduppe.bam: gzip compressed data, extra field
samtools-0.1.18/examples/ex1f-rmdupse.bam: gzip compressed data, extra field
samtools-0.1.18/examples/ex1_sorted.bam: gzip compressed data, extra field
samtools-0.1.18/examples/toy.bam: gzip compressed data, extra field
\end{lstlisting}




\subsubsection{STDIN (standard input) can be the input source}
\begin{lstlisting}
$ find samtools-0.1.18/examples/ -name "*.bam" -type f | parallel file
\end{lstlisting}
output:
\begin{lstlisting}
samtools-0.1.18/examples/ex1a.bam: gzip compressed data, extra field
samtools-0.1.18/examples/ex1.bam: gzip compressed data, extra field
samtools-0.1.18/examples/ex1b.bam: gzip compressed data, extra field
samtools-0.1.18/examples/ex1f.bam: gzip compressed data, extra field
samtools-0.1.18/examples/ex1f-rmduppe.bam: gzip compressed data, extra field
samtools-0.1.18/examples/ex1f-rmdupse.bam: gzip compressed data, extra field
samtools-0.1.18/examples/ex1_sorted.bam: gzip compressed data, extra field
samtools-0.1.18/examples/toy.bam: gzip compressed data, extra field
\end{lstlisting}


Another example: indexing sorted sorted \bam{} files with \samtools{}:
\begin{lstlisting}
$ find dir1 -name "*.bam" | grep sorted |\
	parallel -a -  'samtools index '
\end{lstlisting}
or , without '-a -'
\begin{lstlisting}
$ find dir1 -name "*.bam" | grep sorted |\
	parallel   'samtools index '
\end{lstlisting}

\subsection{Multiple input source}

%%
%% $ for L in 01 02 03 04 05 06 07 08 09 ; do curl -s "http://hgdownload.cse.ucsc.edu/goldenPath/hg19/database/kgXref.txt.gz" | gunzip -c | cut -d '   ' -f 4 | grep _ | uniq | head -n 10 | shuf | head -n 5 | sort > list_genes_${L}.txt  ; done
%%

\begin{lstlisting}
$  parallel echo ::: A T G C ::: A T G C
\end{lstlisting}
output:
\begin{lstlisting}
A A
A T
A G
A C
T A
T T
T G
T C
G A
G T
G G
G C
C A
C T
C G
C C
\end{lstlisting}

\subsubsection{If one of the input sources is too short, its values will wrap}
\begin{lstlisting}
$  parallel echo ::: A T G C N ::: A T G C
\end{lstlisting}

output:
\begin{lstlisting}
A A
A T
A G
A C
T A
T T
T G
T C
G A
G T
G G
G C
C A
C T
C G
C C
N A
N T
N G
N C
\end{lstlisting}

\subsubsection{The input sources can be files}
\begin{lstlisting}
$  parallel  -a <(echo -e "A\nT\nG\nC") -a   <(echo -e "a\nt\ng\nc") echo
\end{lstlisting}
output:
\begin{lstlisting}
A a
A t
A g
A c
T a
T t
T g
T c
G a
G t
G g
G c
C a
C t
C g
C c
\end{lstlisting}


\subsubsection{STDIN can be one of the input sources using '-'}
\begin{lstlisting}
$  echo -e "A\nT\nG\nC" |\
	parallel  -a - -a   <(echo -e "a\nt\ng\nc") echo
\end{lstlisting}
output:
\begin{lstlisting}
A a
A t
A g
A c
T a
T t
T g
T c
G a
G t
G g
G c
C a
C t
C g
C c
\end{lstlisting}

\subsubsection{Instead of -a files can be given after '\texttt{::::}'}
\begin{lstlisting}
parallel  echo :::: <(echo -e "A\nT\nG\nC") ::::   <(echo -e "a\nt\ng\nc")
\end{lstlisting}
output:
\begin{lstlisting}
A a
A t
A g
A c
T a
T t
T g
T c
G a
G g
G t
G c
C a
C t
C g
C c
\end{lstlisting}

\subsubsection{'\texttt{:::}' and '\texttt{::::}' can be mixed:}
\begin{lstlisting}
$ parallel 'grep -Hn ' ::: B7ZGX9 I7FC33 :::: <(ls list_genes_*.txt)
\end{lstlisting}
output:
\begin{lstlisting}
list_genes_01.txt:1:B7ZGX9_HUMAN
list_genes_02.txt:1:B7ZGX9_HUMAN
list_genes_04.txt:1:B7ZGX9_HUMAN
list_genes_07.txt:1:B7ZGX9_HUMAN
list_genes_08.txt:1:B7ZGX9_HUMAN
list_genes_09.txt:1:B7ZGX9_HUMAN
list_genes_02.txt:3:I7FC33_HUMAN
list_genes_06.txt:3:I7FC33_HUMAN
list_genes_07.txt:4:I7FC33_HUMAN
list_genes_08.txt:3:I7FC33_HUMAN
\end{lstlisting}

\subsection{Matching arguments from all input sources}
\subsubsection{With \texttt{--xapply} you can get one argument from each input source}

with \texttt{--xapply}
\begin{lstlisting}
parallel --xapply echo ::: A T G C ::: a t g c n
\end{lstlisting}
output:
\begin{lstlisting}
A a
T t
G g
C c
A n
\end{lstlisting}

without \texttt{--xapply}:
\begin{lstlisting}
$ parallel echo ::: A T G C ::: a t g c n
\end{lstlisting}
output:
\begin{lstlisting}
A a
A t
A g
A c
A n
T a
T t
T g
T c
T n
G a
G t
G g
G c
G n
C a
C t
C g
C c
C n
\end{lstlisting}

\subsubsection{If one of the input sources is too short, its values will wrap}
\begin{lstlisting}
$ parallel --xapply echo ::: A T  ::: g c n
\end{lstlisting}
output:
\begin{lstlisting}
A g
T c
A n
\end{lstlisting}

\subsection{Changing the argument separator}
GNU Parallel can use other separators than ::: or ::::. This is typically useful if ::: or :::: is used in the command to run.
\begin{lstlisting}
$parallel --arg-sep yoyo echo yoyo B7ZGX9 I7FC33 yoyo EIF4G1 PABPC1 yoyo B7ZGX9_HUMAN  C9J4L2_HUMAN
\end{lstlisting}
output:
\begin{lstlisting}
B7ZGX9 EIF4G1 B7ZGX9_HUMAN
B7ZGX9 EIF4G1 C9J4L2_HUMAN
B7ZGX9 PABPC1 B7ZGX9_HUMAN
B7ZGX9 PABPC1 C9J4L2_HUMAN
I7FC33 EIF4G1 B7ZGX9_HUMAN
I7FC33 EIF4G1 C9J4L2_HUMAN
I7FC33 PABPC1 B7ZGX9_HUMAN
I7FC33 PABPC1 C9J4L2_HUMAN
\end{lstlisting}
\subsubsection{Changing the argument file separator}
\begin{lstlisting}
$ ls list_genes_0* | parallel --arg-file-sep schtroumph 'grep -nH' ::: B7ZGX9 I7FC33 schtroumph -
\end{lstlisting}
output:
\begin{lstlisting}
list_genes_01.txt:1:B7ZGX9_HUMAN
list_genes_02.txt:1:B7ZGX9_HUMAN
list_genes_04.txt:1:B7ZGX9_HUMAN
list_genes_07.txt:1:B7ZGX9_HUMAN
list_genes_08.txt:1:B7ZGX9_HUMAN
list_genes_09.txt:1:B7ZGX9_HUMAN
list_genes_02.txt:3:I7FC33_HUMAN
list_genes_06.txt:3:I7FC33_HUMAN
list_genes_07.txt:4:I7FC33_HUMAN
list_genes_08.txt:3:I7FC33_HUMAN
\end{lstlisting}
\subsubsection{Changing the argument delimiter}
\begin{lstlisting}
$ $ echo -n "7ZGX9,I7FC33" | \
	parallel -d , 'grep -nH' :::: - :::  list_genes_0*
\end{lstlisting}
output:
\begin{lstlisting}
list_genes_01.txt:1:B7ZGX9_HUMAN
list_genes_02.txt:1:B7ZGX9_HUMAN
list_genes_04.txt:1:B7ZGX9_HUMAN
list_genes_07.txt:1:B7ZGX9_HUMAN
list_genes_08.txt:1:B7ZGX9_HUMAN
list_genes_09.txt:1:B7ZGX9_HUMAN
list_genes_02.txt:3:I7FC33_HUMAN
list_genes_06.txt:3:I7FC33_HUMAN
list_genes_07.txt:4:I7FC33_HUMAN
list_genes_08.txt:3:I7FC33_HUMAN
\end{lstlisting}

\subsubsection{NULL can be given as  \texttt{\textbackslash{}0}}
\begin{lstlisting}
$  echo -n -e "7ZGX9\0I7FC33" |  parallel -d '\0' 'grep -nH' :::: - :::  list_genes_0*
\end{lstlisting}
output:
\begin{lstlisting}
list_genes_01.txt:1:B7ZGX9_HUMAN
list_genes_02.txt:1:B7ZGX9_HUMAN
list_genes_04.txt:1:B7ZGX9_HUMAN
list_genes_07.txt:1:B7ZGX9_HUMAN
list_genes_08.txt:1:B7ZGX9_HUMAN
list_genes_09.txt:1:B7ZGX9_HUMAN
list_genes_02.txt:3:I7FC33_HUMAN
list_genes_06.txt:3:I7FC33_HUMAN
list_genes_07.txt:4:I7FC33_HUMAN
list_genes_08.txt:3:I7FC33_HUMAN
\end{lstlisting}

\subsubsection{A shorthand for '\texttt{-d  \textbackslash{}0}' is \texttt{-0}}
This will often be used to read files from find ... -print0
\begin{lstlisting}
$ find ./ -name "list_genes_0*.txt" -print0 |\
	parallel -0 'grep -nH'  ::: B7ZGX9 I7FC33 :::: -
\end{lstlisting}
output:
\begin{lstlisting}
./list_genes_01.txt:1:B7ZGX9_HUMAN
./list_genes_04.txt:1:B7ZGX9_HUMAN
./list_genes_02.txt:1:B7ZGX9_HUMAN
./list_genes_09.txt:1:B7ZGX9_HUMAN
./list_genes_07.txt:1:B7ZGX9_HUMAN
./list_genes_08.txt:1:B7ZGX9_HUMAN
./list_genes_06.txt:3:I7FC33_HUMAN
./list_genes_02.txt:3:I7FC33_HUMAN
./list_genes_07.txt:4:I7FC33_HUMAN
./list_genes_08.txt:3:I7FC33_HUMAN
\end{lstlisting}

\section{References}
\begin{itemize}
\item{}
\end{itemize}

\end{document}
